\documentclass[10pt,letterpaper]{article}
% The document class is like the old tex version. Page size and font size are defined here. I try to use relative units for everything (ems instead of cms); that way they scale with the fonts. 

%% Begin the preamble file

% Use the subfiles package to enable compiling of a root document and separate child documents
%
% I found this package to be not as useful as I'd hoped. It does not keep the section numbering 
% constant, and does not allow for cross referencing between section in the individual files.
% \usepackage{subfiles}

% Package for some math symbols
\usepackage{amsmath}

% Package to enable quick switching between single, 1.5, and double spacing
\usepackage{setspace}

% More math fonts symbols
\usepackage{amsfonts}
\usepackage{amssymb}
\usepackage{mathabx}

% For graphics support
\usepackage{graphicx}

% This fixes the original latex computer modern font to support things like ligatures. This is unnecessary with the XeLaTeX command structure.
\usepackage{lmodern}

% This package is for better looking table formatting
\usepackage{booktabs}

% Enable multiple row command in the table environment
\usepackage{multirow}

% The package provides commands to change the page layout in the middle of a document
\usepackage{changepage}

% Enable the use of subcaption for the subfigure formatting
\usepackage[labelformat=simple]{subcaption}


% Enable (using XeTeX or XeLaTeX) the use of other fonts. The font must be installed on the system. 
\usepackage{fontspec}
\setmainfont%
%[BoldFont={ArnoPro-Bold}]{ArnoPro-Regular}
[BoldFont={GaramondPremrPro-Smbd}]{GaramondPremrPro}
%{Calluna-Regular}

%% Create some new commands for use later in the document.
%% This makes it easier if the same command is going to be used many times


% Something for the subfigure, not currently used.
\renewcommand\thesubfigure{(\alph{subfigure})}

% Create a horizontal rule for the title page
\newcommand{\HRule}{\rule{\linewidth}{0.1 em}}

% Create a quick way to set up the indentation environment to tab in a section
\newcommand{\ind}[1]{\begin{adjustwidth}{2.5em}{0em} #1 \end{adjustwidth}}

% Another environment for double tabbing
\newcommand{\indd}[1]{\begin{adjustwidth}{5em}{0em} #1 \end{adjustwidth}}

% Packge to define the margins. I kind of like the standards, so I didn't change it. 
%\usepackage[right=2in,left=2in,top=2in,bottom=2in]{geometry}

% Package to include margin notes. This is used in this document for IMPORTANT notes. 
\usepackage{marginnote}
\renewcommand*{\marginfont}{\bfseries}
\renewcommand*{\marginnotevadjust}{5pt}
\reversemarginpar

% Package to enable links and references throughout the resultant pdf. 
% In the options, you can control colors and how the document will look when opened.
\usepackage{color,hyperref}
\definecolor{darkblue}{rgb}{0.0,0.0,0.50}
\definecolor{black}{rgb}{0.0,0.0,0.0}
\hypersetup{colorlinks,
						bookmarksopen=true,
						bookmarksopenlevel=2,
            linkcolor=darkblue,urlcolor=darkblue,
            anchorcolor=darkblue,citecolor=darkblue,
						%pdfstartview={XYZ null null 1.00},
						%breaklinks,
						pdfstartview=FitH
						}
%\usepackage[anythingbreaks]{breakurl}

% This allows you to control the paragraph spacing. 
\usepackage[parfill]{parskip}



% There is a separate print.bat file included with the documentation. Running it will print
% the full manual and each section as a separate pdf. This should be done following
% any changes of the section files.

\begin{document}

% Put the title page and other frontmatter files here
% The \input command takes the other file and just inserts is verbatim. It is different
% from \include because \include does things like generate separate aux files and clearpage commands.
% I'm using the include command so I can use \includeonly to selectively print specific sections

\begin{titlepage}
\begin{center}

\vspace*{10 em}

\textsc{\Huge HFHS-pacsDisplay}

\vspace{2 em}
\HRule
\vspace{1 em}

\begin{minipage}{0.75\textwidth}
\normalsize{Microsoft Windows programs for generating and installing DICOM grayscale look up tables (LUTs). Includes applications to display grayscale test patterns and to perform QC evaluations of calibrated monitors.}
\end{minipage}

\vspace{2 em}

\begin{minipage}{0.75\textwidth}
\normalsize{IMPORTANT: Only use with LUTs intended for the make and model of monitor used on your workstation. The distribution includes a LUT-Library with generic LUTs for various monitors models. A configuration file needs to be edited to load user generated LUTs.}
\end{minipage}

\vspace{1 em}
\HRule
\vspace{1 em}
{\large February 2014}\\

\vfill

\textsc{\Large Michael Flynn}\\
\vspace{1 em}
\normalsize{Contributing Authors}\\
\vspace{0.5 em}
\textsc{
Philip Tchou \qquad Nicholas Bevins
% Add more authors like this. The \qquad puts in a big space between the names.
}


\clearpage

\vspace*{5 em}

\begin{minipage}{0.75\textwidth}
\begin{center}
\normalsize{GENERAL PUBLIC LICENSE:\\
HFHS-pacsDisplay\\
Copyright \copyright 2014  Henry Ford Health System}
\end{center}
\end{minipage}

\vspace{3 em}

\begin{minipage}{0.8\textwidth}
\normalsize{This program is free software; you can redistribute it and/or modify it under the terms of the GNU General Public License as published by the Free Software Foundation; either version 2 of the License, or
(at your option) any later version.
\bigskip

This program is distributed in the hope that it will be useful, but WITHOUT ANY WARRANTY; without even the implied warranty of MERCHANTABILITY or FITNESS FOR A PARTICULAR PURPOSE.  See the GNU General Public License for more details.
\bigskip

You should have received a copy of the GNU General Public License along with this program; if not, write to the Free Software Foundation, Inc., 51 Franklin Street, Fifth Floor, Boston, MA
02110-1301, USA.
\bigskip

The full license can be found in the GNU-GPL.txt document in the
HFHS/pacsDisplay directory.}
\end{minipage}

\vfill

\begin{minipage}{0.75\textwidth}
\begin{center}
\normalsize{CONTACT INFORMATION:\\
Michael Flynn\\
Henry Ford Health System\\
One Ford Place - Suite 2F\\
Detroit, MI 48202\\
\href{mailto:mikef@rad.hfh.edu}{mikef@rad.hfh.edu}}
\end{center}
\end{minipage}

\end{center}
\end{titlepage}

% Create the TOC
\tableofcontents
% Force everything to print and go on to the next page
\clearpage

% Include the section files below

%\documentclass[../readme.tex]{subfiles} 
%\begin{document}

\section{Installation}
% Any time you want to reference a section so it automatically updates, a label must be included. 
\label{sec:install}

HFHS-pacsDisplay may be installed by a simple Microsoft batch file that invokes an installation tcl script. Instructions for installation using this are summarized in section \ref{sec:quickinstall} below. Alternative manual installation methods are described in section \ref{sec:maninstall}. 

No registry entries are made or changed by this install process and no system background services are installed. All adjustments to the display LUTs are made by a call to the graphic driver that is executed for each user upon login. 

\marginnote{IMPORTANT} Once installation is complete, the behavior desired is established by editing a configuration file:

% Use the \ind command to tab in a section. This was the format I settled on after several iterations. 
\ind{
\textnormal{../HFHS/pacsDisplay/LUTs/Current System/configLL.txt}
}

Unless changed on installation, this folder is located in the following directory based on the operating system:

\ind{
Windows 7: \textnormal{C:/Users/Public}\\
Windows XP: \textnormal{C:/Documents and Settings/All Users}
}

If calibration LUTs for the make and model of your monitor(s) are not in the LUT-Library, you will need to use a photometer to measure the uncalibrated luminance response using lumResponse and then generate a calibration LUT using lutGenerate. 

The lumResponse program supports numerous photometers. The currently recommended photometer is the X-Rite i1Display Pro which can be purchased from several suppliers for about \$225 (USD). 

\subsection{Quick Installation}
\label{sec:quickinstall}

\marginnote{IMPORTANT} The installation described below will fully install executable programs and the distributed LUT-library. However, the current system described is distributed with a linear LUT that will not change monitor contrast. To achieve a calibrated display with improved contrast, the installation must be configured with the proper LUT file.

\textbf{Note:} These instructions assume that you have just unzipped the pacsDisplay distribution files and are currently viewing the \textnormal{.../HFHS} directory which includes this document and the pacsDisplay folder. 

Steps for installing the HFHS-pacsDisplay package: 

\paragraph{Step 1} Open the \textnormal{pacsDisplay} folder and run the \textnormal{pacsDisplay\_install.bat} file.

\paragraph{Step 2} Review the terms of the license agreement.

\paragraph{Step 3} Select the installation options you want:
\bigskip

"The default directory for pacsDisplay programs is:'' (Change)

\ind{
Windows 7: \textnormal{C:/Program Files/HFHS/pacsDisplay}\\
Windows XP: \textnormal{C:/Program Files (x86)/HFHS/pacsDisplay}
}

It is recommended that the default directory for installation of the pacsDisplay files be used. Shortcuts will always be installed to specifc folders in the Start Menu regardless of how this option is set. However, the shortcut targets will need to be changed if a non-default directory is specified.
\bigskip

"The default directory for the LUTs folder is:'' (Change)

\ind{
Windows 7: \textnormal{C:/Users/Public/HFHS/pacsDisplay}\\
Windows XP: \textnormal{C:/Documents and Settings/All Users/HFHS/pacsDisplay}
}

It is recommended that the default directory for the LUTs folder be used. The directory selected is written to the \textnormal{LUTsDir.txt} file in the programs installation directory. If the LUTs folder is manually moved, the \textnormal{LUTsDIR.txt} file must be edited.
\bigskip

Options:
\begin{enumerate}
\item "Overwrite any previous installation?'' (Yes/No)

\textbf{Default:} Yes

This option should be set to "Yes'' if you want to overwrite a previous pacsDisplay installation. A "No" response will abort installation if a prior version is encountered.

\item "Overwrite an existing LUTs directory?'' (Yes/No)

\textbf{Default:} Yes

Answering "Yes'' to this option will install a new LUTs directory along with a default \textnormal{configLL.txt} file. Typically the new release LUTs file will be installed and any user \textnormal{configLL.txt} files or LUT files will be backed up prior to installation.

\item "Install grayscale calibration toolset?'' (Yes/No)

\textbf{Default:} Yes

Answering "Yes'' to this option will install shortcuts to the Start Menu for the various utilities that come with the pacsDisplay program. These tools are intended for IT/physics support.

Answering "No'' will install only the enterprise shortcuts, which provide tools for applying and verifying the calibration. Note that only the shortcuts are not installed, the programs will still be installed in the program files directory.

\item "Run the config file builder (LLconfig) after install?'' (Yes/No)

\textbf{Default:} No

When set to "Yes'', this option will run the LLconfig program after installation. LLconfig is a tool to help build configuration files for pacsDisplay applications. It is recommended that you not use this option unless you are familiar with LLconfig. Instructions for its use can be found in the pacsDisplay directory in the Readme file \textnormal{README-HFHS\_pacsDisplay.txt}. There will be an option to view the Readme file at the end of the install process.
\end{enumerate}

\paragraph{Step 4} Review your selections and press the "INSTALL" button when ready.

\paragraph{Step 5} If the installation completes successfully, an option will be given to view the Readme file \textnormal{README-HFHS\_pacsDisplay.txt} for further details and instructions. 

\subsection{Manual Installation}
\label{sec:maninstall}

The following instructions are intended for manual installation. The files and directories involved only need to be placed in the required locations. No registry entries need to be made or changed. 

This package is intended for installation on a system having a \textnormal{C:/Program Files} directory, or \textnormal{C:/Program Files (x86)} for Windows 7 64 bit systems. 

Unzip the \textnormal{HFHS-pacsDisplay.zip} file to \textnormal{C:/Program Files}. You should see the following path: 

\ind{
\textnormal{C:/Program Files/HFHS/pacsDisplay/..}
}

\marginnote{IMPORTANT} Do not change the organization of files under this path. 

In the above path is a 'Links' directory. It contains separate directories with shortcuts for 32 bit and 64 bit system: 

\ind{
\textnormal{Links/32b\_W7-XP/shortcuts} or \textnormal{Links/64b\_W7/shortcuts}
}

Within these are shortcuts for 

\ind{
\textnormal{allUsers\_startMenu}\\
\textnormal{allUsers\_startMenu\_programs}\\
\textnormal{allUsers\_startMenu\_programs\_startup}
}

For Windows XP systems, these are placed in 

\ind{
\textnormal{C:/Documents and Settings/All Users/...}
}

For Windows 7 systems, these are placed in 

\ind{
\textnormal{C:/ProgramData/Microsoft/Windows/Start Menu/...}
}

where '\textnormal{...}' depends on the name of the source directory. Each of these directories is named so as to indicate where their contents should be copied to. 

There are three locations indicated in the directory names: 

\begin{itemize}
\item \textnormal{allUsers\_startMenu}: \textnormal{/Start Menu} A single shortcut to iQC that places an icon at the top of the Start=>Programs menu that starts the test pattern application. 

\item \textnormal{allUsers\_startMenu\_programs\_startup}: \textnormal{/Start Menu/Programs/Startup} A single shortcut to loadLUT-dcm that will load a DICOM grayscale LUT to the graphic card whenever a user logs into the system. 

\item \textnormal{allUsers\_startMenu\_programs}: \textnormal{/Start Menu/Programs} There are two folders for which one or both can be moved to the appropriate directory: 

\begin{itemize}
\item HFHS ePACS grayscale\\
Two shortcuts to start the test pattern application and to start an application that allows the user to change between a DICOM grayscale and a LINEAR grayscale. 

\item HFHS Grayscale Tools\\
A full set of shortcuts to applications for measuring the luminance response, generating calibrated LUTs, and installing LUTs. This should only be installed for qualified users. 
\end{itemize}
\end{itemize}

It is possible to manually install the application on a drive other than \textnormal{C:/Program Files/...} If this is done, the shortcut targets summarized above need to be modified for the correct path. Each shortcut must have the 'read-only' state removed, the paths modified including the icon paths, and the shortcut saved. It is suggested that the shortcuts be set to 'read-only' after these changes are made. 

%\end{document}
%\documentclass[../readme.tex]{subfiles} 
%\begin{document}

\section{Application Summary}
\label{sec:appsummary}

Below are brief descriptions of each of the applications included in the pacsDisplay package. Further details on these programs and how to use them can be found in the Luminance Response Measurements section. 

\subsection{HFHS ePACS Grayscale (Base applications)}
\label{sec:baseapps}

Shortcuts to the following programs are found in the Windows start programs folder named \textnormal{HFHS ePACS Grayscale}. These are the only installed shortcuts when the Installation option 3 for the toolset is set to NO. 

iQC
\begin{itemize}
\item Presents an image quality test pattern that demonstrates the grayscale characteristics of a display. 
\item The base image size is 756 wide by 792 high. 
\item The window size can be made smaller than the image, in which case the image can be panned using either the scroll bars or 'click and drag' using the left mouse button. 
\item An alternate image of 2X size can be loaded using the <UP> arrow key. The window size will stay the same and the image can be panned by using the mouse and the left button. The regular size image is then obtained using the <DOWN> arrow key. 
\item An alternative window size of 2X that displays the 2X image can be obtained using the <RIGHT> arrow key. The normal window size is obtained using the <LEFT> arrow key.
\end{itemize} 

ChangeLUT
\begin{itemize}
\item Presents a small window with button to load the DICOM Grayscale currently configured or an alternative LINEAR Grayscale. 
\item When used with the iQC test pattern displayed, this provides an effective way to see the difference in image appearance between the normally loaded grayscale (LINEAR) and the DICOM grayscale. 
\end{itemize}

Additionally, a shortcut is put in the Windows startup directory where it is executed every time a user logs in to the workstation

loadLUT-dicom
\begin{itemize}
\item Put in the appropriate startup directory (different for W7 and XP) 
\item Loads the DICOM grayscale as configured in .../LUTs/Current System.
\end{itemize} 

\subsection{HFHS Grayscale Tools (additional applications)}
\label{sec:addapps}

Shortcuts to the following additional programs are found in the Windows start programs folder named \textnormal{HFHS Grayscale Tools}. These are installed when the Installation option 3 for the toolset is set to YES. 

gtest
\begin{itemize}
\item Application to present uniform regions with adjustable gray levels to support macro images of lcd pixel structures. All controls are implemented with key bindings. Help information is shown using the ? button.
\end{itemize}

i1meter
\begin{itemize}
\item Application to display live readings of either chromaticity (u', v') or D65 distance (Cuv) and either luminance (cd/m$^2$) or illuminance (lux).
\item Currently, only the i1DisplayPro meter is supported.
\item Help information and additional functionality is shown using the ? button.
\end{itemize}

iQC-2x
\begin{itemize}
\item Same as iQC but opens in the 2X image/window size.
\end{itemize}

lumResponse
\begin{itemize}
\item Application to generate a test pattern that steps through a palette of gray levels so that luminance can be measured using an IL1700 luminance meter connected using a serial line interface. 
\item This is used in 256 mode to measure the calibrated response of a monitor having DICOM grayscale LUTs installed. 
\item This is used in 766 or 1786 mode to measure the uncalibrated response of a mono\-chrome or color display having a LINEAR LUT installed. 
\item Make, model, and serial number of display normally used for identification. 
\item A plot of the luminance vs p-values is provided at the end of a measurement. 
\item Support a QC mode for verification of the luminance response of a calibrated monitor.
\end{itemize}

LutGenerate
\begin{itemize} 
\item Generates a LUT for installation with loadLUT. 
\item Reads luminance response measured by lumResponse. 
\item Requires specification of maximum luminance, luminance ratio and ambient luminance.
\end{itemize}

uLRstats
\begin{itemize}
\item A utility tool to select a set of uncalibrated luminance response (uLR) files (i.e. palette file) and generate an average uLR file from which a generic LUT can be generated for model specific use.
\end{itemize}

Uninstall pacsDisplay
\begin{itemize}
\item Uninstall program that removes all files and folders from the \textnormal{../HFHS/pacsDisplay} directory. 
\item An option is provided to save the LUTs directory.
\end{itemize}

Shortcuts to these utility programs are placed in the \textnormal{loadLUT utilities} folder within the \textnormal{HFHS Grayscale Tools} folder.

LLconfig
\begin{itemize}
\item This program provides a form-fillable interface to help build a config file for LoadLUT. 
\item Supports up to 4 displays and provides access to the model and serial number information from the EDID.
\end{itemize} 

loadLUT-dicom, loadLUT-linear
\begin{itemize}
\item Depending on the argument in the shortcut, this loads the LUTs in the \textnormal{../LUTs/Current System} or the \textnormal{../LUTs/Current System/linear} directory. \textnormal{LoadLUT.exe} is executed from the \textnormal{execLoadLUT.exe} program to report errors. The LUT is loaded in the driver with no application window. 
\item The error messages reported can be changed in the execLL-messages.txt file in the loadLUT directory. 
\end{itemize}

loadLUT
\begin{itemize}
\item Provides a utility tool to find a LUT file and load it to a specific monitor number.
\end{itemize}

%\end{document}
%\documentclass[../readme.tex]{subfiles} 
%\begin{document}

\section{Luminance Response Measurements}
\label{sec:lumresponsemeasurements}

The lumResponse application can be used to measure the gray palette for a display and record the luminance values in a uLR text file (Palette modes). These uLR palettes are then used to generate a DICOM calibration LUTs as described in the next section. LumResponse can also be used to measure the luminance response of a calibrated monitor and record the values in a cLR text file (QC modes). The cLR values can then be evaluated in relation to the DICOM GSDF. 

\subsection{LumResponse}

This program puts a large secondary window on the screen with a gray background and a central square target region. The target gray level is varied to assess display luminance response. The gray intensity of the target region is cycled through increasing intensity values to measure luminance versus gray value. The measurements are made by luminance meter (photometers or colorimeters), that are typically connected using a USB port. Currently, four meter types are supported. 

\subsubsection{Mode Options}

The 'Mode' button at the top left of the utility selects the type of measurement to be made. The current mode is displayed in the window title. The various modes are as follows: 

For uLR measurement: 

\textbf{1786 Palette} This mode is used for measuring the uncalibrated luminance response of a color display. Make sure the display is using a linear LUT (i.e - no calibration) before proceeding. 

\textbf{766 Palette} This mode is used for measuring the uncalibrated luminance response of a mono\-chrome display. Make sure the display is using a linear LUT (i.e - no calibration) before proceeding. 

For cLR measurement: 

\textbf{QC (256x1)} This mode is used for measuring the luminance response of a display after a LUT has been installed in order to verify that a DICOM calibration is in place. One measurement is made for 256 driving levels. 

\textbf{QC (16x2)} This mode is used for measuring the luminance response of a display after a LUT has been installed in order to verify that a DICOM calibration is in place. Two measurements are made at each of 16 different primary driving levels (8,24,40,...248). The measurements are made a driving levels just above and just below the primary driving level from which the contrast is computed. For some luminance meters, color gray tracking is also reported. 

Other modes: 

\textbf{DEMO MODE} This mode is for demonstration purposes. It will quickly cycle from black to white but no measurements will be taken. The IL1700 does not need to be connected for the demo to run. 

\textbf{OTHER} This option allows the user to use a customized perturbation series for the luminance measurements. For advanced users only. 

\subsubsection{Geometry and Meter Options}

The "GEOM" and "METER" buttons provide access to advanced options for the test image window and the photometer, respectively. The "GEOM" options can be used to customize the geometry and position of the test image. The initial values for the geometry options are set in the LRconfig.txt file which is in the \textnormal{.../pacsDisplay-BIN/lumResponse/} directory of Program Files [(x86)]. The initial configuration is for a 1024x1024 window with an 8 cm gray target region on monitors with a .230 mm pixel pitch. This should be suitable for the majority of monitor models. 

The "METER" options include selection of the meter model and settings effecting the way display luminance is measured. Further information is given by the "?" buttons next to each setting. 

Four meters are currently supported: 

\textbf{International Light IL1700 Research Radiometer} This device uses a serial line interface to report measured values. It has been replaced by the ILT1700 that uses a USB interface. The IL1700 option is retained only for legacy purposes. 

\textbf{IBA Dosimetry LXcan Spot Luminance Meter} This device is capable of making spot luminance measures some distance from the monitor surface. It is effective in measuring the amient luminance which is used in building a calibration LUT. A USB driver provided by IBA Dosimetry is used to record values. 

and two meters that use the Argyll CMS spotread procedure (www.argyllcms.com):

\textbf{X-Rite i1Display 2 photometer} This modestly priced photometer uses a USB interface. The photometer used the USB device driver from the open source Argyll Color Management System (Argyll CMS, www.argyll.com) which must be loaded the first time the meter is used. (see \href{www.argyllcms.com/doc/Installing_MSWindows.html}{www.argyllcms.com/doc/Installing\_MSWindows.html}) While many of these are still in use, it has been replaced by the i1Display Pro described next. 

\textbf{X-Rite i1Display Pro} This recently introduced and modestly priced photometer from X-Rite has improved precision and faster response than the i1Display 2. It is the currently recommended device if you need to buy a photometer and is available from several distributors for about \$200-250 USD. It is also sold with different labels such as the NEC SpectraSenor Pro. The device uses communicates using USB ports as a human interface device (HID) using the Windows HID driver. As such, it will generally attach and be ready to use without the need to load a driver.

\subsubsection{Making Measurements} 

Once a mode has been selected and the meter specified (defaults to i1Display Pro), a Display ID should be created to identify the measurement results once they are saved. The name that is selected will be used to name the output file. 

The "Display \#'' box is used to select the display that you wish to identify. This number is the same as that used by Windows and listed in the Display Properties window. Pressing the "GET EDID ID'' button will then retrieve the model and serial number (S/N) information from the EDID for that display and use it to build the Display ID. The result is displayed in the text box. A custom ID can be entered into the text box if so desired. It is suggested that the model and serial number of the display be included. 

The remaining portion of the lumResponse window provides the steps for the luminance response measurement and a button to begin each step. They are described here: 

\paragraph{Step 1: Position Test Image} This opens up a test image window that needs to be centered on the screen of the display to be measured. The window should be about the same size as the screen. The display dize can be changed manually using the GEOM settings. 

\paragraph{Step 2: Initialize Meter} This button begins communication with the luminance meter. It activates the meter display at the bottom of the window, providing information regarding the measurement. The details of this monitor are explained further below.

At this point, the luminance meter should be setup in front of the display, and centered on the square target in the middle of the test window. Depending on the meter used, it should be positioned close to or in contact with the display surface such that the luminance is not perturbed. the room lights should be set low and if necessary the monitor should be covered with a dark cloth. If used, the cloth must not cover any vents in the back of the display as this can cause it to heat up quickly and may affect the measurement. 

\paragraph{Step 3: Record Data} This starts the measurement process. You will be asked to be sure that the monitor is in the correct calibration (linear for uLRs and DICOM for cLRs) and that the power saver settings will not cause the screen to darken. Pressing this button once a measurement has begun will pause the measurement and provide an option to abort or continue. 

\paragraph{Step 4: Save Data Press} this button to save the luminance data once a measurement is complete. You will be asked where to save the output file. The default directory in the \_NEW folder in the LUT-Library. An option to plot the luminance vs p-values is provided after the data is saved.
%Should this last step be updated to include what is plotted for the QC run? 

Columns for the luminance response saved file:

%\begin{center}
%\begin{tabular}{rp{.6\textwidth}}
%\toprule
%MEASUREMENT No. & Sequential measurment number.\\
%\\
%AVG LUMINANCE & The average luminance value measured for each gray step.\\
%\\
%RGB VALUE & Octal RGB value for the measurement.\\
%\\
%GRAY-STEP & The current stage of the luminance measurement. This number represents the standard graylevel (1-256). Negative values represent the initialization frames.\\
%\\
%SUB-STEP & This second number represents the perturbation steps in the graylevel sequence. There are 7 sub- steps in 1786 mode and 3 in 766 mode. The 256 and Demo modes do not use sub-steps.\\
%\\
%dL/L & Relative difference between the current and the prior measurement.\\
%\\
%x & Chromaticity x value (for supported meters, Yxy, CIE 1932)\\
%\\
%y & Chromaticity y value (for supported meters, Yxy, CIE 1932)\\
%\bottomrule
%\end{tabular}
%\end{center}

\begin{itemize}
	\item[] \textbf{MEASUREMENT No.} \enskip Sequential measurement number.
	\item[] \textbf{AVG LUMINANCE} \enskip The average luminance value measured for each gray step.
	\item[] \textbf{RGB VALUE} \enskip Octal RGB value for the measurement.
	\item[] \textbf{GRAY-STEP} \enskip The current stage of the luminance measurement. This number represents the standard graylevel (1-256). Negative values represent the initialization frames.
	\item[] \textbf{SUB-STEP} \enskip This second number represents the perturbation steps in the graylevel sequence. There are 7 sub- steps in 1786 mode and 3 in 766 mode. The 256 and Demo modes do not use sub-steps.
	\item[] \textbf{dL/L} \enskip Relative difference between the current and the prior measurement.
	\item[] \textbf{x} \enskip Chromaticity x value (for supported meters, Yxy, CIE 1932)
	\item[] \textbf{y} \enskip Chromaticity y value (for supported meters, Yxy, CIE 1932)
\end{itemize}

Files are save with files names of:
\ind{\textnormal{uLR\_MANF\_MODEL\_SN.txt} for uncalibrated palette files and\\
\textnormal{cLR\_MANF\_MODEL\_SN.txt} for calibrated QC files.} 

The luminance response recorded with lumResponse represent surface luminance in the absence of the luminance caused by reflected room lights, Lamb. When doing QC evaluations or computing DICOM calibration LUTs, a value of Lamb is specified and added to each value. 

\subsection{QC Evaluations}

The American Association of Physicists in Medicine (AAPM) described a quantitative test of a DICOM calibrated monitor in 2005 (AAPM On Line report \#3). This was included in an IEC standard as a basic test (IEC). 

\subsubsection{QC (16x2) Evaluation}

The lumResponse applications includes routines to evaluate the QC 16x2 cLR. The AAPM and IEC method makes 18 luminance measurements at equally spaced gray levels (Digital Driving Levels, DDL). For a graphics system with 256 gray levels, these are spaced every 15 gray levels (17 x 15 + 1 = 256). 17 relative luminance changes are then evaluated and this contrast is compared to the DICOM GSDF. The ACR-AAPM-SIIM Technical Standard for Electronic Imaging (2012) recommends
\begin{quote}
"The contrast response of monitors used for diagnostic interpretation should be within 10\% of the GSDF over the full LR. For other uses, the contrast response should be within 20\% of the GSDF over the full LR." 
\end{quote}

The measurement protocol used by lumResponse for the QC 16x2 mode is essentially the same as the AAPM and IEC method except that 16 pairs of measures are made over the full range of a 256 level grayscale. The base value for each of the 16 measurement pairs increments in steps of 16; 0,16,32,...,240. The two measures are then made by adding 5 or 11 to the base value with the red, green, and blue values of each being equal. 

The contrast evaluated from these measure represents the contrast estimate, dL/L, for levels of 8, 24, 30, 46, ..., 248 with the contrast computed for gray level changes of 6. This provides an improved estimate of contrast compared to measures using gray level changes of 15 for the AAPM and IEC method. Additionally, the protocol makes three measurements of Lmin at a gray level of 0 and one measurement of Lmax at a gray level of 255. 

\textbf{Note:} The protocol is set in the 16phase.txt file located in the following folder in Program Files [(x86)]:
\begin{center}
	\textnormal{.../pacsDisplay-BIN/lumResponse/}
\end{center}


After saving the QC 16x2 cLR in a directory with the monitor model name, the user has the option to evaluate the results. An evaluation report, \textnormal{QC-lr.txt}, is place in the same folder along with four graphic plots in png format. The gnuplot command file is along left in the folder, \textnormal{QC-plot.gpl}. The plotted results include: 

\ind{
\textbf{QC-Plot-LUM.png} \enskip Luminance vs Gray Level using a semi log plot

\textbf{QC-Plot-dLL.png} \enskip Contrast, dL/L, vs Gray Level with 10\% and 20\% error conditions. The maximum relative error along with L'max and L'min are labeled.

\textbf{QC-Plot-JND.png} \enskip JNDs per Gray Level vs Gray Level (see DICOM 3.14)

\textbf{QC-Plot-uv.png} \enskip Color gray tracking with u'V' relative to D65.
}

The evaluation uses a default value of Lamb from the 'METER' options. The results can be re-evaluated with a different Lamb by using the QC check application that select the cLR file to re-evaluate and provides an entry for the new Lamb value.

\subsubsection{QC (256x1) Evaluation}

AAPM OR3 also described an advanced measurement test for which the calibrated luminance is measure for each of 256 gray levels and the contrast or JNDs per gray level evaluated using the difference between each measurement as adjacent gray levels. The QC (256x1) mode measures each of 256 luminances with red, green, and blue values being equal (i.e. 256 gray levels). These are then evaluated using a method similar to that for the QC (16x2) mode.


%\end{document}
%\documentclass[../readme.tex]{subfiles} 
%\begin{document}

\section{Calibration LUTs, lutGenerate}
\label{sec:lutcal}

Calibration of a monitor using pacsDisplay involves loading a look-up table (LUT) to the graphic driver. The LUT is a list of 256 RGB values used to replace the standard grayscale values (R=G=B) in order to match with the DICOM grayscale display function (GSDF). Creating the LUT requires measuring the full range of gray values that the display is capable of, the grayscale palette, and then using those values that are closest to the desired DICOM GSDF in a calibration look-up table (LUT). LutGenerate takes a uLR file (output by LumResponse) and builds a LUT file based on the parameters you specify. 

The basic usage steps are:

\paragraph{Step 1: Select a uLR File} Start by pressing the "SELECT FILE" button and choosing the appropriate uLR file for the display to be calibrated. LutGenerate will read the uLR file and automatically update the fields throughout the window as appropriate. This includes the 'Desired Maximum Luminance' field, which will be set to the maximum luminance value found in the uLR file. Both the display name and desired maximum luminance may be changed manually after loading a uLR. 

\paragraph{Step 2. Determine Calibration Parameters} Three parameters must be specified before generating the calibration LUT:

\ind{
\textbf{Lamb} \enskip The ambient luminance expected for the display. This is typically measure using a photometer when the monitor is turned off and the room lighting is set to establish the desired illumination. Room lights should not cause direct specular reflections on the monitor surface.

\textbf{L'max} \enskip The desired actual maximum luminance, which is the maximum from the monitor plus the ambient luminance(Lmax + Lamb) can be changed. The initial value is set at the maximum from the selected uLR file plus the entered Lamb. If changed it must be lower that shown initially.

\textbf{r'} \enskip The luminance ratio equal to L'max/L'min. ACR-AAPM-SIIM Technical Guidelines recommend a value of 350.
}
\bigskip

Typically these are left the value of L'max taken from the uLR, and the default r' of 350. Lamb may need to be adjusted (along with the room lighting) in order to maintain Ar within an acceptable range. 

\paragraph{Step 3. Verify Calibration Parameters} Hitting the 'ENTER' key after changing one of the calibration parameters will update the other values presented below: 

\ind{
\textbf{Ar, Lamb/Lmin (Ambient Ratio)} \enskip The AAPM TG-18 report calls for this value to be no move that 2/3 and recommends a value of less than 1/4. The text changes to yellow if the value is above 1/4 but less than 2/3 and red if it is above 2/3 

\textbf{Target Lmax} \enskip This value is equal to the maximum luminance that will result from loading the generated LUT in the graphic driver in the absence of ambient luminance (i.e. Lmax not L'max). The text will turn red if it goes above the possible maximum luminance indicated by the uLR. 

\textbf{Target Lmin} \enskip This value is equal to the minimum luminance that will result from loading the generated LUT in the graphic driver in the absence of ambient luminance (i.e. Lmin not L'min). The text will turn red if it goes below the possible maximum luminance indicated by the uLR. 

\textbf{Possible Lmin} \enskip The minimum luminance found in the selected uLR file. 

\textbf{Possible Lmax} \enskip The maximum luminance found in the selected uLR file.
}

Check to be certain that these values are correct before continuing.

\paragraph{Step 4. Generate the LUT} Click on the "GENERATE" button to build the calibration LUT. You will be asked where to save the LUT file. The are commonly stored within the LUT-Library/\_NEW directory in a folder specific to the workstation monitor. 

%\end{document}
%\documentclass[../readme.tex]{subfiles} 
%\begin{document}

\section{Loading LUTs, loadLUT}

Usage: loadLUT.exe [(working directory)] 

Command Line Options: (working directory) - If a working directory is specified, then all input files will be read from that directory. The log file will also be written to that directory. If no directory is specified, the current directory will be used. 

loadLUT is the program in the pacsDisplay package responsible for applying a calibration LUT to a display. It reads from a configuration file, \textnormal{configLL.txt}, which it looks for in its starting directory or working directory (if specified). ChangeLUT, execLoadLUT, and loadLUTdemo provide expanded interfaces for running loadLUT. Further details on using loadLUT and related utilities are given in this section. 

\subsection{The Configuration File}

loadLUT uses a configuration file, \textnormal{configLL.txt}, to designate which displays are to be calibrated and what LUTs to load. It will look for this file either in the same directory as loadLUT.exe or in a specified directory, as mentioned in the usage instructions above. 

For a standard pacsDisplay installation, there are \textnormal{configLL.txt} files in two strategic directories that need to be properly configured relative to the make and model of each monitor installed on the system: 

\ind{
\textnormal{.../LUTs/Current System}\\
\textnormal{.../LUTs/Current System/Linear}
}

where \textnormal{"...''} refers to the LUTs installation path.

Below is the standard layout for the configLL.txt file, configured for a two-monitor system: 

\begin{center}
\begin{tabular}{ll}
\multicolumn{2}{l}{\# First 2 lines reserved for comments.} \\
\# & \\
/LUTsearch [dir] & Optional line, invokes model/SN LUT search \\
/LDTsearch & Optional line, invokes dated LUT search \\
/noload & Optional line, checks LUT but does not load \\
/noEDID & Optional line, bypasses EDID checks \\
/nolog & Optional line, prevents writing of log file \\
2 & Number of displays to be calibrated \\
1 & Display number \\
"MANF\_MODEL" & Model descriptor (or "*") \\
"SN" & Serial number (or "*") \\
"calLUT1.txt" & Default calibration filename \\
%\cmidrule(r){2-2}
2 & $\Lsh$\\ 
"MANF\_MODEL" & Four lines for \\
"SN" & each of N monitors \\
"calLUT2.txt" & \rotatebox[origin=c]{180}{$\Rsh$}\\
%\cmidrule(r){2-2}
\end{tabular}
\end{center}

Comments: The first two lines are reserved for comments and will not affect how loadLUT performs.

\textbf{Options:} Options, if present, are included right after the comment lines and must start with a '/' character. The following options are currently available: 

\textbf{/LUTsearch [dir]}

\ind{This option is used when managing a group of systems having monitor models that are in the LUT-Library. The argument, dir, is the LUT-Library full path name. During installation, a default \textnormal{configLL.txt} is created in the Current System directory which has the /LUTsearch option with the correct pathname based on the installation. 

\textbf{Note 1:} The path should be entering using quotes. If not, paths with spaces will not be valid.

\textbf{Note 2:} The directory argument can be formed with either forward or reverse slash characters.

\textbf{Note 3:} If no argument is provided, loadLUT will assume the the LUT-Library is in the 'program file' installation directory which was used in early versions. Beginning with package 5A, this is no longer valid and a correct directory should always be entered. 

When this option is set, loadLUT first gets the monitor manufacturer (MANF), model (MODEL), and serial number (SN) from the EDID that Windows store in the registry. If it doesn't find them, it will use the the config file values. If the /noEDID option is set, then the model descriptor and serial number are also taken from the config file. 

The program then looks in the LUT-Library to find the appropriate LUT folder based on the model descriptor, \textnormal{.../LUT-Library/<MANF\_MODEL*>/}. The "*" denotes a wildcard. If the folder name has characters beyond <MANF\_MODEL>, it will still be accepted. 

Several loadLUT utilities can be used to identify the model descriptor, MANF\_MODEL, for the monitors on a particular system. The first row of the monitor table created by EDIDprofile has the model descriptor. LLconfig will also get the EDID and show the model descriptor. 

The program will first look in the LUTs directory of the LUT folder,

\ind{
\textnormal{.../LUT\--Library/<MANF\_MODEL*>/LUTs},
}

to see if a LUT is present with a matching serial number. These LUTs are generated by LUTgenerate with a file name of the form \textnormal{LUT\_MANF\_MODEL\_SN\_*}. Additional values at the end document LUTgenerate input parameter values. SN can be either the 4 digit VESA EDID number or the extended VESA EDID number (see the configLL program).

If a serial number specific LUT is not found, loadLUT will look in the LUT folder directory for a generic LUT file, \textnormal{LUT\_MANF\_MODEL\_GENERIC\_*}. If that also fails, it will load the default LUT file specified in \textnormal{configLL.txt}. This default LUT file must be in the Current System directory.}

\textbf{/LDTsearch [\#]}

\ind{This rarely used option is similar to the LUTsearch option. It uses the year and week of manufacture, along with the model name from the EDID to find the appropriate LUT file from the \textnormal{../Current System/LDT/} directory. The "\#" indicates the date tolerance, i.e. - the number of weeks before or after the specified date that the search will accept. The default is 3 weeks. 

If the /noEDID option is set, then the search fails. 

file format: LDT\_<MANF\_MODEL>\_<year(xxxx)><week(1-52)>\_* 

\textbf{Note:} If both LUTsearch and LDTsearch are set, LUTsearch takes priority. If LUTsearch fails, loadLUT will still run LDTsearch. If LDTsearch also fails, then loadLUT will use the default LUT file.}

\textbf{/noload}

\ind{When this option is set, loadLUT will save the current display LUTs in backup files without loading new LUTs.}

\textbf{/noEDID}

\ind{This option prevents loadLUT from searching the registry for EDID information. This may allow loadLUT to avoid errors with some display configurations, but also disables loadLUT's ability to verify display information.}

\textbf{/nolog}

\ind{When this option is set, loadLUT will not attempt to generate a log file. This may be needed if loadLUT is run under an account that has limited access privileges.}

\textbf{Number of Displays:} Following the options is the number of displays to be calibrated. Each display is listed below this line and four lines must be present for each display. 

\textbf{Display Number:} The reference to display number is for the number that pacsDisplay finds in the registry for the monitor using getEDID. The Display Number is usually the same that Windows reports for each display under Display Properties $\rightarrow$ Settings. However, for systems that have certain remote management software installed the numbers can be shifted up for some of monitors in a multi-monitor system. If this occurs, used EDIDprofile to document the getEDID display number which is in the first row of the table. 

\textbf{Model Descriptor (or "*"):} The model descriptor is checked against what is listed in the EDID. A mismatch will cause loadLUT to output an error message and will not load a LUT. It will then continue on to the next display. The Model can be replaced with a "*" in order to bypass the check for that line. If there are spaces in the model descriptor, it should be enclosed in quotes. This is generally recommended even if there are no spaces. 

\textbf{Serial number (or "*"):} The serial number is checked against what is listed in the EDID. A mismatch will cause loadLUT to output an error message and will not load a LUT. It will then continue on to the next display. The Model can be replaced with a "*" in order to bypass the check for that line. If there are spaces in the serial number, it should be enclosed in quotes. This is generally recommended even if there are no spaces. 

In general, the EDID will contain a 4-digit serial number. Some EDIDs also include an extended serial number longer than 4 digits. If either one matches the serial number given in the config file, then this check will be successful. 

\textbf{Default Calibration Filename:} The next line is the default calibration filename. This is the LUT file that the program will use to adjust the display. If any of the search options are in place, then they will take precedence in selecting a LUT file. Should the search options be unsuccessful, then loadLUT will use this filename by default.

Several files are installed in the Current System directory some of which must be present:

\ind{
The Current System folder contains the \textnormal{configLL.txt} file that should be edited based on user requirements along with the linear LUT, \textnormal{linearLUT.txt}, used as the default LUT. For routine system management, it is recommended that a LUT folder be built in the LUT-Library and the linear LUT left as the default. 

The Linear folder, \textnormal{.../LUTs/Current System/Linear}, contains a \textnormal{configLL.txt} files for loading linear LUTs and a copy of the linear LUT file, \textnormal{linearLUT.txt}.
}

For this distribution, the LUT-library contains collections of uncalibrated luminance response files, uLR files, along with derived average response and calibration LUT files for these monitors: MANF\_MODEL\_(3g) where the number in parentheses indicates the number of uLR files. The "g" indicates that a generic file also exists. Additionally, directories with uLR files but no generic LUT are included for other monitor models. 

For this distribution, the Current System LUT directory has a default \textnormal{configLL.txt} file that asserts the /LUTsearch option with the model\_name and S/N set to "*". If an S/N match or GENERIC match is not found in the LUT-library, the default LUT is assigned to that for a linear LUT. To demonstrate that calibrated LUTs are being loaded on a monitor that does not have a LUT folder in the LUT-Library, this linear LUT needs to be temporarily replaced with some calibration LUT. 

\marginnote{IMPORTANT} The 'Current System' \textnormal{configLL.txt} file must be properly configured for the monitors used as is illustrated in the examples below.

\subsection{Sample Config Files}

\textbf{Example 1:} This configuration is for a single display, identified in Windows as display "1", with no model name or S/N verification. 

\indd{
\# First 2 lines reserved for comments.\\
\# \\
1 \\
1 \\
"*" \\
"*" \\
"<LUT filename for display \#1>"
}


\textbf{Example 2:} This configuration extends to display "3". The EDID information for this display will be checked for a matching model descriptor. If the model name given here is different from what is found in the EDID, an error will occur and the program will abort. No check will be made to match a serial number. 

\indd{
\# First 2 lines reserved for comments. \\
\# \\
2 \\
1 \\
"*" \\
"*" \\
"<LUT filename for display \#1>" \\
3 \\
"DELL 2007FP" \\
"*" \\
"<LUT filename for display \#3>"
}

\textbf{Example 3:} This configuration includes the option to search for the LUT files in the \textnormal{../LUTs/LUT-Library} directory. For display \#1, the filename will be based on the model name and serial number given in the EDID. For display \#3, the model name "DELL 2007FP" will be checked against that in the EDID, while the serial number will be whatever is found in the EDID. These model names and serial numbers will be used to build the filenames for LUTsearch. If the search for the specific files and generic files are unsuccessful, then the listed default LUT files will be used instead. 

\indd{
\# First 2 lines reserved for comments. \\
\# \\
/LUTsearch \\
2 \\
1 \\
"*" \\
"*" \\
"<LUT filename for display \#1>" \\
3 \\
"DELL 2007FP" \\
"*" \\
"<LUT filename for display \#3>" 
}


\textbf{Example 4:} This configuration includes a search based on the date of manufacture of the display. It will search through the files in \textnormal{../Current System/LDT/}, choosing the one that is closest to the date of manufacture, but not going beyond 52 weeks from that date. 

\indd{
\# First 2 lines reserved for comments. \\
\# \\
/LDTsearch \\
52 \\
2 \\
1 \\
"*" \\
"*" \\
"<LUT filename for display \#1>" \\
3 "DELL 2007FP" \\
"*" \\
"<LUT filename for display \#3>"
}

\textbf{Example 5:} Here we have a 4 display system with LUTs being drawn from the \textnormal{../LUTs/LUT-Library} directory. However, since the /noEDID option is being used, LUTsearch will instead build the intended filenames using the monitor descriptors and serial numbers given below. Those that have only "*" for both the model and serial number will automatically fail the file search and will instead default to the given LUT filename. Display \#3 gives a model name, but no serial number. LUTsearch will not be able to build a specific LUT filename for display \#3, but it will still search for a generic file in the <model\_name*> directory before going to the default LUT file. 

\indd{
\# First 2 lines reserved for comments. \\
\# \\
/noEDID \\
/LUTsearch \\
4 \\
1 \\
"*" \\
"*" \\
"<LUT filename for display \#1>" \\
2 \\
"*" \\
"*" \\
"<LUT filename for display \#2>" \\
3 \\
"DELL 2007FP" \\
"*" \\
"<LUT filename for display \#3>" \\
4 \\
"DELL 2007FP" \\
"T61164A5ABYU" \\
"<LUT filename for display \#4>"
}

\subsection{LLconfig Tool}

LLconfig is a tool to help build a LoadLUT configuration file for a particular display setup. The format for the configuration file is described in the \textnormal{README-HFHS\_pacsDisplay.txt} document in the pacsDisplay directory. 

\marginnote{IMPORTANT} The EDID functions require that \textnormal{getEDID.exe} be in the same directory as the LLconfig executable. 

\textbf{USAGE:}

\begin{enumerate}
	\item \textbf{Options} The top bar lists the possible LoadLUT options that can be set. Click on the boxes next to the option names to select (or de-select) them.
	\item \textbf{Number of Displays} Selecting a number means that entries for all of the displays up to and including that number will be included. For example, if "3" is selected, entries for displays 1, 2, and 3 will be generated in the final configuration file. These display numbers correspond to the numbers in the "Display Properties" window.
	\item \textbf{Display Entries} Text entry bars are provided to list the configuration information for each individual display.
	\begin{itemize}
		\item[] \textbf{Monitor Descriptor} This is the name of the display as given by the EDID information. If left blank, a wild card character ("*") will be inserted into the configuration file instead.
		\item[] \textbf{Serial Number} This is either the 4-digit or extended serial number found in the EDID. If left blank, a wild card character ("*") will be inserted into the configuration file instead. 
		\item[] \textbf{Default LUT file} This is the LUT file that will be applied to the display if LUTsearch and LDTsearch are not selected or if the file being searched for is not found. If left blank, \textnormal{linearLUT.txt} will be inserted as the LUT file for that display.
		\item[] \textbf{GET EDID} This button runs getEDID and looks for the monitor descriptor and serial number entries for that display. If found, these items will be copied to the appropriate text entry bars. If both a 4-digit and extended serial number are found, then the user will be asked to select one.
		\item[] \textbf{SELECT LUT} This button opens a file-select window so that the user can browse for the LUT file they want for that display.
	\end{itemize}
	\item \textbf{BUILD CONFIG FILE} This button will take the current form information and build a configLL.txt file formatted for LoadLUT. The user will be asked where to save the file.
	\item \textbf{RESET FORM} This button resets the form to its initial state.
	\item \textbf{QUIT} This button exits the program. Form information will not be saved.
\end{enumerate}

%\end{document}
%\documentclass[../readme.tex]{subfiles} 
%\begin{document}

\section{The LUT-Library}
\label{sec:lutlibrary}

Section \ref{sec:lutcal} above explains how the loadLUT program can search the LUT-Library to find a appropriate LUT. Using the LUTsearch option, the LUT-Library is searched to find a LUT matching the manufacturer, model, and serial number obtained from the EDID data for each monitor found. In not found, loadLUT looks for a generic LUT. 

This section explains how the LUT-Library is organized and how generic LUTs can be built using the uLRstats tool.

\subsection{LUT-Library directories}

The LUT-library contains a collection of uncalibrated luminance response files (uLR files) and calibration look-up table files (LUT files). The files are organized in directories with names based on the model descriptor (manufacturer\_model) normally encountered when reading the EDID from a monitor device. The number in parenthesis indicates the number of monitors having uLR files within the directory. The letter g after the parenthetic number indicates that a generic lut file has been prepared and is included. 

Within each monitor directory, for example \textnormal{DELL\_2007FP\_(10g)}, There are two subdirectories call uLRs and LUTs. Typically a number of monitors will be evaluated using lumResponse to obtain uLRs with filenames that begin with uLR and contain the manufacturer, model, and serial number. For example,
\ind{
\textnormal{uLR\_DELL\_2007FP\_C953667D38RL.txt}\\
in \textnormal{.../LUTs/LUT-Library/DELL\_2007FP\_(10g)/uLRs/}
}
The uLRs directory may also contain plot results from uLRstats which is explained in the section \ref{sec:genericluts}. 

Calibration look-up tables files built using lutGenerate are stored in the LUTs directory. For example,
\ind{
\textnormal{LUT\_DELL\_2007FP\_C953667D38RL\_0.10\_217.00\_350.00.txt}\\
in \textnormal{.../LUTs/LUT-Library/DELL\_2007FP\_(10g)/LUTs/}
}
These files will begin with LUT, will contain the manufacturer, model, and serial number, and will have the values for Lamb, Lmax, and Luminance ratio that were used to build the LUT. If multiple LUT files are present, as may be the case if several are generated with different luminance ratio values, the <S/N> or GENERIC expression can be appended with \_* characters. In this case the file used under the /LUTsearch option will be the first found with alphabetic listing. 

The monitor directory may also contain a generic LUT built using uLRstats and a README file with information explaining how and why the monitor was added to the LUT-Library. For example,
\ind{
\textnormal{LUT\_DELL\_2007FP\_GENERIC\_0.18\_227.90\_350.00.txt}\\
in \textnormal{.../LUTs/LUT-Library/DELL\_2007FP\_(10g)}
}
The serial number section of the filename is replaced with 'GENERIC' for these files. 

Since loadLUT searchs the LUT-library as a part of the /LUTsearch option, the location of the library and the library structure should be maintained as installed. 

The file names for LUT files should be of the form
\ind{
LUT\_<model\_name>\_<S/N>\_*
}
where <S/N> can be either the 4 digit VESA EDID number or the extended VESA EDID number and * can be any addition characters The model\_name is that returned in the VESA EDID as the model descriptor and is commonly of the form "DELL 2007FP" with the spaces replaced by '\_' characters. If a generic lut is desired it should be selected or built using the uLRstats utility and placed at the top of the model name directory in the format 
\ind{
LUT\_<model\_name>\_GENERIC* 
}

If a new monitor folder is built, email the pacDisplay contact person identified at the beginning of the file to make arrangements for inclusion of the new monitors in the next LUT-Library release. A log file is maintained in the LUT-Library folder documenting the history of contributed monitor folders. The LUT-Library folder also has a file documenting the current version number, VERSION\_INFO.txt. The version information is read and shown using the ChangeLUT program. Beginning with version pacsDisplay version 5A, the LUT-Library version is maintained separate from the program installation package version. 

\subsection{uLRstats (generic LUTs)}
\label{sec:genericluts}

Experience has shown that monitors with the same manufacturer and model will have an uncalibrated response file that is similar except for brightness variations. This results if a manufacturer changes the model designation when the LCD panel used for a product line is changed. In this case, a generic LUT can be used for DICOM calibration without having to measure the S/N specific uLR and create an S/N specific LUT. 

uLRstats is a tool to evaluate the uLRs of a set of monitors, identify those to used as a basis of a generic uLR, and produce a file with the average luminance for each palette entry. This can then be used with lutGenerate to build the generic LUT file. 

When uLRstats is first run, use the SELECT button to select the directory with a set of uLRs to be evaluated. Within the Select window, first use the Path button to select the uLR directory from the Folder Browse window. The window should open with the \_NEW directory of the LUT-Library where new monitor folders can be built. Closing the window with OK enters the selected folder path in the filePath entry. Assuming the uLRs all have filenames of the form uLR*.txt, use the Glob button to build a list of filenames, then highlight the files to be evaluated using SHIFT+<LEFT CLICK> and CTRL+<LEFT CLICK>. After files are selected, close the Select window with OK. The processing bar at the bottom of the application window will change to indicated the number of uLRs to analyze. Use this button to begin. 

During the analysis, the uLRs are checked for unusually large changes in luminance, dL/L. These are recorded in the log file. A dialogue window with report the total number and offer to open the log file. It is not unusual to have a modest number of out of limit changes. 

The analysis will create six files in the uLR directory:

\ind{
\textbf{uLR\_DELL\_2007FP\_GENERIC.txt} \enskip The average uLR

\textbf{uLR-LminLmax} \enskip A table of Lmin, Lmax, and filename for each uLR processed. This can be useful when evaluating which files may not want to be included in the generic uLR. The order of these files corresponds to the number in the plots.

\textbf{uLR-plot.gpl} \enskip Gnuplot command file to generate three plots shown in a window and saved as png files.
 
\textbf{uLR-Plot\_LminLmax.png} \enskip Plot of Lmax vs Lmin

\textbf{uLR-Plot-dL\_L.png} \enskip Plot of the dL/L values for each palette entry and file.

\textbf{uLR-PlotULRs.png} \enskip Plot of luminance vs palette entry for each file.
}

Typically, uLRs are evaluated in two steps. First all uLRs are evaluated. Those monitors with atypical uLRs are then identified from the plots. Then a second analysis is done using only typical uLRs. 

When done, a generic LUT can be built with lutGenerate.

%\end{document}
%\documentclass[../readme.tex]{subfiles} 
%\begin{document}

\section{Monitor EDIDs}

Monitors need to communicate information regarding the device to the graphics card so that appropriate display communication can be established. The Video Electronics Standards Association (VESA) standardized the method to do this called the Extended Display Identification Data (EDID). The monitor device information is stored in Read Only Memory (ROM) within the monitor where it can be read by the graphic card. The EDID data structure contains information including the manufacturer, the model, the serial number, the data of manufacture. Display information includes the display size, array size and timings support for other array sizes. For further information see, \href{http://en.wikipedia.org/wiki/Extended_display_identification_data}{http://en.wikipedia.org/wiki/EDID}

Various pacDisplay programs obtain EDID information from the Windows Registry to identify the monitors present on a particular workstation. This is done within loadLUT when LUT-Library searches are done to find LUTs. It is also done with lumResponse and lutGenerate to establish a display ID used to build luminance response files and LUT files. A utility tool, edidProfile, is used to quickly obtain monitor ID information for all displays on a workstation. 

\subsection{getEDID}

The command line program \textnormal{getEDID.exe} in the distributed package is used by several pacsDisplay programs to get information from the EDID. It takes a single argument that is the display number. The program is a core utility that is not supported by a graphic application other than EDIDprofile described in the next section. 

The program uses an approach suggested by Calvin Guan, a software engineer at ATI Technologies Inc., that was posted on the internet in 2004. A series of function calls from the Windows graphics device interface GDI) library for C++ is used to get the data from the monitor’s EDID which is decoded to obtain details on the display. Visual C++ was used to build getEDID. 

\subsection{EDIDprofile}

When evaluating a display workstation, particularly those with multiple monitors, it is useful to be able to tabulate information from the EDIDs that document the workstation characteristics. This is particularly true when doing a QC evaluation of a system. 

EDIDprofile is a utility program that sequentially executes getEDID from monitor device 1 to N. N is set in a configuration file to a value of 6.The program first obtains information from the registry regarding the workstation and it's processors. Then the EDID packets obtained are parsed to obtain:

\begin{itemize}
\item Adapter display ID
\item Adapter string
\item Monitor Descriptor
\item Extended S/N
\item Week of manufacture
\item Year of manufacture
\item Max. horizontal image size (mm)
\item Max. vertical image size (mm)
\item Horizontal array size: Native
\item Horizontal array size: Current
\item Vertical array size: Native
\item Vertical array size: Current
\item Est. hor. pixel size (microns)
\item Est. ver. pixel size (microns)
\end{itemize}

The estimated pixel sizes are computer from the image size and array resolution. 

The current and native array sizes are included since the current should always be set to the native for a digital monitor. The pixels size is computed from the image size in the EDID and is thus subject to a small error relative to manufacturers specifications. This estimate is still useful in establishing that the pixel size meets professional recommendations. 

The results of EDIDprofile are recorded in a text file that can be written in a user selectable directory. The default is the \_NEW directory in which a folder is created with the workstation name. This can subsequently be used to record QC results using lumResponse.

%\end{document}

\end{document}